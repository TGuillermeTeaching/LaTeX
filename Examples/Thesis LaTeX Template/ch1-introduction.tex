\chapter{Introduction}
\label{chap:introduction}%Note this label will be used to refer to the chapter throughout. So if you change the order of chapters it still knows this one is this file, but can call it chapter 1 or 2 or whatever depending on the order. S oti's better than calling it chapter 1.

\begin{quoteshrink}
  ``Really grandiose sounding quotes from Darwin always make a thesis feel more professional''
  \hfill{Natalie Cooper, p.~15}
\end{quoteshrink}

\noindent
Here is the introduction to the thesis.

\section{Subsection of my intro}

Here I introduce some concepts.

%The nice thing with LaTeX is you can leave notes in here, and/or stuff that you might include
%but might not. Rather than deleting it just comment out with %
%Should save time converting from thesis to papers and vice versa.

\section{Another subsection of my intro}

Here I introduce other concepts and provide a figure.

\begin{figure} %note figure will appear where LaTeX thinks it fits best. See http://en.wikibooks.org/wiki/LaTeX/Floats,_Figures_and_Captions if you need to tell it where to put the figure.
  \centering
  \includegraphics[width = 30cm, height = 10cm, keepaspectratio=true]{ch1-introduction/Happy_smiley_face.png}
  \caption[Happy face because I finished]%this is what appears in table of contents
  {A nice happy face for you to enjoy.}%this is under the figure
\label{fig:happy}
\end{figure}

\section{Structure \& contents of this thesis}
In this thesis, I do some really cool stuff.
%
In \chapref{labstudy}, I do some lab work.
%
In \chapref{velociraptor}, I train a velociraptor to ride a hoverboard.
%
Finally, in \chapref{conclusions}, I close with a discussion of the
limitations of the methods used in the thesis, and suggest some future
directions.

