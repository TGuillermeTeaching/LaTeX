\documentclass{article} % Setting the class of the document

\begin{document} % starting the document

%This will be ignored.

This will not be ignored.

\% This won't be ignored either. % Note the backslash that ignores the character (ignore ignore!)

Also, 




%intro



note
that
it doesn'
t
matter                   whether
I
    format that really weirdly.
% To start a new line, just add an empty line

new line!
% Or you can use commands like 
\newline
That also creates a new line
%Commands are useful formating "tags" that allow to modify the raw text.
\newline
Commands always start with backslash ($\backslash$) %Note that backslash is a special character for starting commands.
%So if I actually want to print this character, I have to use a command.

%Because it's a command for a special character, I have to "protect it" between "$" signs.
Some commands don't need an input like $\backslash$noindent % note that quotes are specific here (double opening `` and double closing '')

\noindent that removes the indent at the start of a new line (like $\backslash$newline).


Some other commands take some input like the commands for \textit{italicizing} and \textbf{bolding} the text.
\section{Or even some other that can bring a bit of structure:}
\section{Some structure is needed!}
We covered the followings:

\begin{enumerate} %And finally there exists some commands that begin/end some other ones (like in this case "enumerate")
\item Ignored comments with \%
\item The unimportance of the text format (focus on the content!)
\item \textbf{Some} \textit{basic} \texttt{commands} %with \texttt (3 t's at the end) that puts the text in courier font.
\end{enumerate}

\subsection*{Now try to play around with all that and restructure all this} % \subsection creates a new hierarchical level and * just after tells latex to ignore the numbering (skip) of this new level


\end{document} % Ending the document (don't forget this one!)