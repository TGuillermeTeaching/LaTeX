\documentclass[a4paper,11pt]{article}

\usepackage{natbib}
\usepackage{enumerate}
\usepackage[osf]{mathpazo}
\usepackage{lastpage}
\usepackage{url}
\usepackage{hyperref}
\pagenumbering{arabic}
\linespread{1.66}

\begin{document}

\begin{flushright}
Version dated: \today
\end{flushright}
\begin{center}

%Title
\noindent{\Large{\bf{Getting started with \LaTeX}}}\\
\bigskip
%Author
Thomas Guillerme\\\href{mailto:t.guillerme@imperial.ac.uk}{t.guillerme@imperial.ac.uk}

\end{center}

\section{Install your favourite text editor}
You can use pretty much any text editor depending on your machine and your preferences.
Be aware though that often, some plugins/packages are necessary to make LaTeX run smoothly.

There are many like the universal multiplatform \href{http://texstudio.sourceforge.net/}{\textbf{TeXStudio}}, \href{http://www.tug.org/mactex/index.html}{\textbf{MacTeX}} for Mac or \href{http://miktex.org/}{\textbf{MikTeX}} for Windows (just google \texttt{LaTeX editor}).
However, I personally prefer \textbf{Sublime Text 2} that is multi-platform, way less RAM and memory demanding and does combine really well with some other languages you are working with (\texttt{R} for example!).

\section{Installing LaTeX on Sublime Text 2}
\subsection{Getting the softwares}
\begin{itemize}
\item \textbf{Sublime Text 2}: the text editor (\url{http://www.sublimetext.com/2}).
\item \textbf{A pdf viewer}: I suggest \href{http://skim-app.sourceforge.net/}{\textbf{Skim}} for Mac or \href{http://www.sumatrapdfreader.org/download-free-pdf-viewer.html}{\textbf{Sumatra}} for Windows.
\end{itemize}

\textbf{WARNING:} if you're using MacOS 10.11 (El Capitain), you will have to install \href{http://www.tug.org/mactex/index.html}{\textbf{MacTeX}} prior to installing LaTeX on sublime \textbf{Sublime Text 2}.

\subsection{Installing LaTeX on Sublime}
\begin{itemize}
\item Open \texttt{Sublime Text 2}.
\item Copy past the relevant code from \href{https://packagecontrol.io/}{\textbf{package control} (click on \texttt{download}) in the console (in \texttt{View} menu, select \texttt{Show Console})}.
\item Go in the \texttt{Tools} menu and select \texttt{Command Palette...}.
\item Type in \texttt{Package Control: Install Package} and press \texttt{Enter}.
\item Type in \texttt{LaTeXTools} and press \texttt{Enter} (\href{https://github.com/SublimeText/LaTeXTools/blob/master/README.markdown}{more info on LaTeX tools}). Note that \texttt{LaTeXing} works fine as well but can be a bit annoying if you want to integrate LaTeX in more advanced coding (e.g. \texttt{Sweave} or \texttt{knitr}).
\end{itemize}

\subsection{Synchronising Skim}
Open Skim then go into the \texttt{Preferences} menu, then \texttt{Sync} and set \texttt{preset} to \texttt{Sublime Text 2}.

\section{Some useful links}
\begin{itemize}
\item \url{https://en.wikibooks.org/wiki/LaTeX}
\item \url{http://google.com/}
\end{itemize}
And that covers pretty much everything!
LaTeX is a language used and polished since the last 30 years so there is always an easy answer to your \href{http://lmgtfy.com/?q=ask+a+question+in+latex}{questions}!

\section{I don't want to learn \LaTeX because:}
\subsection{Word is more intuitive}
This is simply not true.
Because you are used to one software doesn't make it more intuitive!
You're just good at using it!
And is Word really intuitive?
Thing about doing specific tasks like formatting tables or using overall numbered footnotes references without new lines between each footnote.
\subsection{I'm not good at coding}
Unless you're creating templates \LaTeX \ is not really coding.
Think of it as a version of Word where you don't need to use your mouse: for example, when writing the title of the introduction part, rather than typing \texttt{introduction}, clicking on it and then clicking on the format button and then clicking on the desired format, you just write \texttt{\\section\{\}} in front of \texttt{introduction}...
\subsection{My supervisor/collaborator doesn't want to use \LaTeX}
Your supervisor does not need to learn \LaTeX\ (although they should!): you can just send them the compiled pdf.
Although you won't get the ``fancy looking'' comments and track changes that word uses, they can jut comment the pdf (in the same way it's done so many times when reviewing papers!)...
\subsection{I just like to find excuses not to learn cool tools that will help me to work more efficiently}
Fine. That is actually a more realistic argument. I stop here.


\end{document}
