\documentclass{article}\usepackage[]{graphicx}\usepackage[]{color}
%% maxwidth is the original width if it is less than linewidth
%% otherwise use linewidth (to make sure the graphics do not exceed the margin)
\makeatletter
\def\maxwidth{ %
  \ifdim\Gin@nat@width>\linewidth
    \linewidth
  \else
    \Gin@nat@width
  \fi
}
\makeatother

\definecolor{fgcolor}{rgb}{0.345, 0.345, 0.345}
\newcommand{\hlnum}[1]{\textcolor[rgb]{0.686,0.059,0.569}{#1}}%
\newcommand{\hlstr}[1]{\textcolor[rgb]{0.192,0.494,0.8}{#1}}%
\newcommand{\hlcom}[1]{\textcolor[rgb]{0.678,0.584,0.686}{\textit{#1}}}%
\newcommand{\hlopt}[1]{\textcolor[rgb]{0,0,0}{#1}}%
\newcommand{\hlstd}[1]{\textcolor[rgb]{0.345,0.345,0.345}{#1}}%
\newcommand{\hlkwa}[1]{\textcolor[rgb]{0.161,0.373,0.58}{\textbf{#1}}}%
\newcommand{\hlkwb}[1]{\textcolor[rgb]{0.69,0.353,0.396}{#1}}%
\newcommand{\hlkwc}[1]{\textcolor[rgb]{0.333,0.667,0.333}{#1}}%
\newcommand{\hlkwd}[1]{\textcolor[rgb]{0.737,0.353,0.396}{\textbf{#1}}}%

\usepackage{framed}
\makeatletter
\newenvironment{kframe}{%
 \def\at@end@of@kframe{}%
 \ifinner\ifhmode%
  \def\at@end@of@kframe{\end{minipage}}%
  \begin{minipage}{\columnwidth}%
 \fi\fi%
 \def\FrameCommand##1{\hskip\@totalleftmargin \hskip-\fboxsep
 \colorbox{shadecolor}{##1}\hskip-\fboxsep
     % There is no \\@totalrightmargin, so:
     \hskip-\linewidth \hskip-\@totalleftmargin \hskip\columnwidth}%
 \MakeFramed {\advance\hsize-\width
   \@totalleftmargin\z@ \linewidth\hsize
   \@setminipage}}%
 {\par\unskip\endMakeFramed%
 \at@end@of@kframe}
\makeatother

\definecolor{shadecolor}{rgb}{.97, .97, .97}
\definecolor{messagecolor}{rgb}{0, 0, 0}
\definecolor{warningcolor}{rgb}{1, 0, 1}
\definecolor{errorcolor}{rgb}{1, 0, 0}
\newenvironment{knitrout}{}{} % an empty environment to be redefined in TeX

\usepackage{alltt}
\usepackage[sc]{mathpazo}
\usepackage[T1]{fontenc}
\usepackage{geometry}
\geometry{verbose,tmargin=2.5cm,bmargin=2.5cm,lmargin=2.5cm,rmargin=2.5cm} % Settings for sweave
\setcounter{secnumdepth}{2} % Settings for sweave
\setcounter{tocdepth}{2} % Settings for sweave
\usepackage{url}
\usepackage[unicode=true,pdfusetitle, % Settings for sweave
 bookmarks=true,bookmarksnumbered=true,bookmarksopen=true,bookmarksopenlevel=2, % Settings for sweave
 breaklinks=false,pdfborder={0 0 1},backref=false,colorlinks=false] % Settings for sweave
 {hyperref} % Settings for sweave
\hypersetup{ % Settings for sweave
 pdfstartview={XYZ null null 1}} % Settings for sweave
\IfFileExists{upquote.sty}{\usepackage{upquote}}{}
\begin{document}
 % Settings for sweave:


\title{LaTeX using sweave}

\author{Thomas Guillerme\\t.guillerme@imperial.ac.uk}


\maketitle

This is a really quick demo to show some integration between LaTeX and \texttt{R} via \href{http://yihui.name/knitr/}{knitr} that allows to include everything (the text, the \texttt{R} code and the results) in a single document using \textbf{Sweave} language (.Rnw).

\section{Introduction}

Let's start with a first sentence. And then get some cites to support it \cite{Cooper2008,Brazeau2011,harrisonamong-character2014}.

\section{Material and Methods}
\subsection{Material}
I got my data from thin air.

\begin{knitrout}
\definecolor{shadecolor}{rgb}{0.969, 0.969, 0.969}\color{fgcolor}\begin{kframe}
\begin{alltt}
\hlcom{## Generating some data}
\hlstd{data1} \hlkwb{<-} \hlkwd{rnorm}\hlstd{(}\hlnum{100}\hlstd{)}
\hlstd{data2} \hlkwb{<-} \hlkwd{rnorm}\hlstd{(}\hlnum{100}\hlstd{)}
\end{alltt}
\end{kframe}
\end{knitrout}

\subsection{Methods}
\label{methods}
\texttt{R} is awesome for doing some analysis \cite{R}.
And it can use pretty sophisticate equations like the Mind Blowing Metric Quotient (MBMQ):
  \begin{equation}
  \label{equation}
        MBMQ=\frac{\sum{observations}}{n}
  \end{equation}
where \textit{n} is the total number of \textit{observations}.

\begin{knitrout}
\definecolor{shadecolor}{rgb}{0.969, 0.969, 0.969}\color{fgcolor}\begin{kframe}
\begin{alltt}
\hlcom{## The Mind Blowing Metric Quotient function}
\hlstd{MBMQ} \hlkwb{<-} \hlkwa{function}\hlstd{(}\hlkwc{X}\hlstd{) \{}
    \hlkwd{return}\hlstd{(}\hlkwd{sum}\hlstd{(X)}\hlopt{/}\hlkwd{length}\hlstd{(X))}
\hlstd{\}}
\end{alltt}
\end{kframe}
\end{knitrout}

\section{Results}
Now let's check the results in a table:

\begin{knitrout}
\definecolor{shadecolor}{rgb}{0.969, 0.969, 0.969}\color{fgcolor}\begin{kframe}
\begin{alltt}
\hlcom{## My results in a table}
\hlkwd{matrix}\hlstd{(}\hlkwd{c}\hlstd{(}\hlkwd{MBMQ}\hlstd{(data1),} \hlkwd{MBMQ}\hlstd{(data1)),} \hlkwc{nrow} \hlstd{=} \hlnum{1}\hlstd{,} \hlkwc{ncol} \hlstd{=} \hlnum{2}\hlstd{,} \hlkwc{byrow} \hlstd{=} \hlnum{TRUE}\hlstd{,}
    \hlkwc{dimnames} \hlstd{=} \hlkwd{list}\hlstd{(}\hlkwd{c}\hlstd{(}\hlstr{"Mind Blowing"}\hlstd{),} \hlkwd{c}\hlstd{(}\hlstr{"Data 1"}\hlstd{,} \hlstr{"Data 2"}\hlstd{)))}
\end{alltt}
\begin{verbatim}
##                  Data 1     Data 2
## Mind Blowing 0.02755136 0.02755136
\end{verbatim}
\end{kframe}
\end{knitrout}

Or even with some more fancy analysis and a plot
\begin{knitrout}
\definecolor{shadecolor}{rgb}{0.969, 0.969, 0.969}\color{fgcolor}\begin{kframe}
\begin{alltt}
\hlstd{model} \hlkwb{<-} \hlkwd{summary}\hlstd{(}\hlkwd{lm}\hlstd{(data1} \hlopt{~} \hlstd{data2))}
\hlkwd{plot}\hlstd{(data2, data1,} \hlkwc{pch} \hlstd{=} \hlnum{19}\hlstd{,} \hlkwc{col} \hlstd{=} \hlstr{"grey"}\hlstd{)}
\hlkwd{abline}\hlstd{(model}\hlopt{$}\hlstd{coefficients[[}\hlnum{1}\hlstd{]], model}\hlopt{$}\hlstd{coefficients[[}\hlnum{2}\hlstd{]])}
\end{alltt}
\end{kframe}

{\centering \includegraphics[width=.8\linewidth]{figure/minimal-lm-1} 

}



\end{knitrout}

This model doesn't fit super well, it has a \textit{R$^2$} of 0.013882!

\bibliographystyle{vancouver}
\bibliography{References}

\end{document}
